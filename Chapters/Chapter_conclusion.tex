\chapter{Conclusion} % Main chapter title

\label{chapter:conclusion} 

This dissertation describes three in-game element recommendation systems (i.e., starting items, characters, and opponents, respectively) designed to apply in the pre-match stage to indirectly or directly improve player engagement based on theoretical foundations or data-driven approaches. They answered the fundamental research question: \textit{How can we design in-game element recommendation systems working in the pre-match stage, which can improve player engagement in match-based video games?} The overall contribution of my dissertation is the enrichment of arsenal of recommendation techniques that recommend  pre-match in-game elements which are either winning-effective or positively influential in player engagement measured  quantitatively.

The starting item and character recommendation systems, within the context of one-vs-one and team-vs-team settings respectively, attempted to identify winning-effective in-game elements efficiently from a large amount of candidates. Their usefulness is based on the assumption that presenting winning-effective in-game elements to players could improve their competence, and theoretical foundations such as Self-Determination Theory and Flow that improving player competence leads to better engagement. Thus, the starting item and character recommendation systems can be seen as an indirect approach to improve player engagement. We conducted experiments and showed that our proposed systems are able to recommend equally or more winning-effective in-game elements than previous methods with computational resources efficient enough for large-scale or real-time usage. 

The third system aims to directly optimize player engagement through matchmaking, i.e., recommendations of opponents. We defined player engagement quantitatively as churn risk and rely on a data-driven approach to determine the recommendation of opponents that is optimal for  the engagement metric. As far as we know, this is the first system that has formally treated matchmaking as an optimization problem for player engagement quantitatively measured. We built a simulated system using real game data, showing significant advantages of the proposed matchmaking system in retaining players over existing methods.


% The starting item and character recommendation systems manifest how match outcomes can be swayed positively from one side of player(s) in match-based video games. They are seen as indirect means of improving player engagement because the trigger of recommendations may need to be decided by other systems, which we did not investigate further in this thesis. Exemplifying how and how much match outcomes should be influenced for a population of players, the opponent recommendation is seen as a direct means of improving player engagement because the system directly optimizes the objective function in which expected match outcomes after opponent matching maximizes the all participated opponents. 

