% Chapter Template

\chapter{Introduction} % Main chapter title

\label{chapter:intro} % Change X to a consecutive number; for referencing this chapter elsewhere, use \ref{ChapterX}

%----------------------------------------------------------------------------------------
%	SECTION 1
%----------------------------------------------------------------------------------------
The world has witnessed a collection of video games that manage to make people addicted into playing day and night. The series of \textit{Super Mario Bros.} (Nintendo Co. Ltd) have sold over 160 million since the 1980s \cite{mariosale}. Players enjoy the excitement and fulfillment of passing through game levels designed with various challenges. \textit{League of Legends}, a 5-vs-5 online competitive game, has 90 million users registered, 27 million unique daily players and 7.5 million concurrent users at peak \cite{lol_fanbase,lol_27million}. The game is characteristized by competition, collaboration and strategies that players are hard to experience in the real world. Recently, \textit{Pok\'{e}mon Go} (Niantic, Inc.) has swept the globe with more than 500 million downloads \cite{pokemongo}, triggering the crowd to walk on the street to interact with the game. At the same time, the viewership of video games has also grown rapidly. e-Sports, a subset of highly competitive video games, was estimated to have hundreds of million viewership in 2017 and projected to still grow at 12\% each year~\cite{superdata2017}. Besides being commonly perceived as a means of entertainment, video games have also been utilized for various purposes: video games have promoted aesthetic appreciation~\cite{jarvinen2008understanding}, health awareness~\cite{shiyko2016effects}, science learning~\cite{cooper2013increasing}, and socialization~\cite{ferguson2013friends}.

As video games continue to penetrate an increasing number of fields for improving the wellness of our society, one question arises naturally: how can we keep player engaged? This question is critical from game designers to game companies.

Based on the number of human player participants, video games can be categorized as single-player and multi-player game. The former puts the single player against pre-programmed challenges or AI-controlled opponents, while the latter allows player interaction with other individuals in partnership, competition or rivalry. Player-versus-Player (PvP) is one mode in which multiple players (who may form as teams) directly engage in competition or combat. PvP games are usually played match after match. A match refers to a stretch of time during which two sides of players (often random online players) compete for a certain goal. In this thesis, I focus on PvP games because they have been increasingly popular in recent years with professional competitions and frenzy communities sweeping over the globe~\cite{superdata2016}, and arguably PvP games have more variety in interactions and strategies, making relevant research more challenging. 

In PvP games, match outcomes have close relationship with player engagement suggested by theoretical and data-driven findings. Player engagement can be understood as "the level of continuation desire experienced in-game, during play or over a longer period of time, when players dedicate themselves to coming back and playing a game again and again" \cite{schoenau2011player}. According to Flow theory~\cite{sweetser2005gameflow,flow1990psychology,chen2007flow}, part of player engagement comes from challenges which match the player's skill level and keep at an appropriate pace. Monotonous match outcomes are therefore not desirable because they indicate consistent over-/under-challenging player experience. Theories have also pointed out the pursuit of victory and achievement as a reason of engagement~\cite{schoenau2011player,yee2006motivations,sherry2006video,wu2010falling}. This raises the importance to carefully maintain the winning rate for players, especially those inexperienced players who have difficulty of winning a match. 

Although game designers can inject heuristics into game design to influence the winning probability and match outcomes, for instance, certain game equipment boosts are designed to be present only in disadvantageous situations, such methods require heavy domain expertise and lack of variability. The alternative mechanism is personalized recommendation, which suggests adaptive game contents to players as they experience the game. Here, game contents can be skill-competitive opponents~\cite{Delalleau2012,herbrich:trueskill}, preferred play styles~\cite{thue2007interactive,magerko2008intelligent}, effective in-game avatars~\cite{conley2013does}, and so on. A distinctive feature of  personalized recommendation in PvP games is that adjusting one's winning probability will adversely change that of the opponent(s) who are human players too. Therefore, we care about not only how personalized recommendation can control match outcomes for one side of players, but also how and how much match outcomes should be affected to satisfy all participants, to the best extent. This poses a unique challenge different than recommendation systems in movies~\cite{amatriain2012netflix}, e-commerce~\cite{linden2003amazon} and news~\cite{das2007google}, where recommended items to one user are relatively independent to other users' experience.




Generally, personalized recommendation for PvP games can run in either \textit{pre-match stage} or \textit{in-match stage} to control match outcomes. Pre-match stage is the time window from players request to start a match until the match officially begins. Depending on the nature of the game, there are various game elements to be determined before the match can officially starts, such as opponents, game avatars to be played throughout the match, and initial items to bring into the match. In-match stage is the period when the real match takes place and players engage in competition. Various recommendation techniques have been applied in both stages. For example, in the pre-match stage, opponents can be recommended to match the player's skill level in order to make games sufficiently challenging~\cite{sweetser2005gameflow,flow1990psychology,chen2007flow}. In the in-match stage, tactical and strategical hints suitable to players knowledge level~\cite{weber2009data,cunha2014rtsmate} can be recommended in order to help players with difficulty of winning the match and prevent disengagement from frustrating experience~\cite{schoenau2011player}.

% Hence, fairness-based matchmaking systems for opponent recommendation have been thus designed to create competitive matches~\cite{herbrich:trueskill,myslak2014developing}, based on various skill models that accurately estimate player skills~\cite{glickman1999parameter,elo1978rating,herbrich:trueskill}. 


Despite multiple stages where personalized recommendation techniques to improve player engagement are applicable, the scope of my proposal is in \textit{the pre-match stage}. It is the initial, yet fundamental, gateway into the main experience of the match. Bad determination of pre-match elements (e.g., opponents and game avatars) put players into disadvantages from the beginning of matches and could lead to toxic behaviors such as quitting the match early~\cite{shores2014identification} and cyberbullying~\cite{kwak2015exploring}. 




My thesis outlines two personalized recommendation techniques in the pre-match stage: initial item recommendation and opponent recommendation. First, initial items are those game artifacts players designate to bring into the match before it starts. In some games, in-match strategies and play styles largely depend on the choice of the initial items. Thus, initial items could have an important impact on match outcomes. There can be a large amount of choices of initial items from total available items and it is often beyond inexperienced players' ability to make good decision. I propose a personalized initial item recommendation system to find the most competent initial items against a stereotyped opponent (equipped with specific initial items and strategies). The initial item recommendation system exemplifies how match outcomes can be affected positively for one side of a match. Second, I propose an opponent recommendation system in which recommended opponents are predicted to result in match outcomes beneficial to both sides' engagement. Here, engagement is measured by a data-driven approach and will be elaborated later. The opponent recommendation system exemplifies both how and how much match outcomes should be affected for all involved players. We will show that the long-held belief to always create competitive matches (thus hard-to-predict match outcomes) for all players is worth revisiting.

My thesis takes into account three practical reasons to include initial item recommendation and opponent recommendation. First of all, they are selected due to the anticipated large impact on today's digital game landscape. Initial items are commonly seen game elements in PvP games. Taking our testbed \textit{Collectible Card Games} (CCG) for example, digital CCGs like Hearthstone (Blizzard Inc.) reached a record of 40 million registered accounts in 2016~\cite{hearthstonepopular}. Opponent recommendation is designed as a general framework to be applied in comprehensive PvP games. Second, the author spent significant efforts in data collection during his research collaboration and industry internships to make relevant data and software tools available. The difficulty of accessing to sensitive game data or related software tools limits our current research scope to these two feasible topics. Third, the two topics are still in much infancy, showing much room for improvement. The more detailed motivation for each project is laid out as follows. 


First, the initial item recommendation algorithm is to determine effective initial items to maximize a player's win rate against a stereotyped opponent. This stereotyped opponent could be someone whom the player frequently lost to with specific initial items and strategies. I choose \textit{Collectible Card Games} (CCGs) as the testbed to study initial item recommendation. CCGs are 1-vs-1 games, in which initial items refers to \textit{deck}, a fixed number of cards (usually just a few dozens) selected from a pool of  cards (usually a few hundreds). The player's in-game abilities in the upcoming match will rely on the cards in the deck. CCGs feature a striking challenge that the number of choices of deck construction is exponentially large. Deck construction is widely regarded influential on final match outcomes~\cite{garcia2016evolutionary}. There exist many online forums and websites for players to discuss their deck construction strategy (e.g.,~\cite{hearthpwn,icyveins}). Ideally, a competitive deck contains cards which have synergies with each other and strong opposition against the strategies based on the opponent's deck. A challenge is that players must invest a large amount of effort in studying card effects, before which they might have become disengaged because of failing to construct an effective deck. A very limited number of works have been done on this topic~\cite{garcia2016evolutionary}. Garc{\'\i}a-S{\'a}nchez et al.~\cite{garcia2016evolutionary} proposed to use \textit{Evolutionary Algorithm} (EA)~\cite{eiben2003introduction} inspired by natural evolution to search for the most competent deck from repeatedly filtering of randomly modified candidate decks. The filtering criteria is based on the \textit{fitness value} defined as the average win rate of a candidate deck pitting against the stereotyped opponent deck. A default, greedy-based AI was used as a proxy of strategy executor for both sides. However, the limitation of their method is that a deck's strength may not be completely realized by just using the universal, greedy-based AI. In real-world matches, there exist many advanced strategies that the greedy-based AI fails to execute, e.g., wait until two cards are drawn to the hand to make combined play. Moreover, as pointed out by Garc{\'\i}a-S{\'a}nchez et al.~\cite{garcia2016evolutionary}, the evolution of candidate decks is currently totally random rather than following a more effective, human-like deck modification. For example, rather than randomly replacing a card with another card, human players do not break up well-known good card combinations  or replace a card with a similar but much weaker one. The "blindness" of EA hinders its speed to converge to a competitive deck, given the nature of exponentially large choices. Therefore, we propose to improve the effectiveness of the existing recommendation method hence more practical on influencing match outcomes: (1) search decks more intelligently; (2) replace the universal, greedy-based AI with more advanced AI that can realize useful strategies of a given deck. The initial item recommendation system exemplifies \textit{how} match outcomes can be affected positively \textit{from one side}.

\begin{table}
\centering
\caption{
Average churn risks vary drastically upon players' recent three match outcomes (\emph{(W)in}, \emph{(L)ose} or \emph{(D)raw}). Data is from a popular PvP game made by Electronic Arts, Inc. Churn risk is measured by the ratio of the players who stop playing within a period time (7 days in this table) after a match. The churn risk of some states with repeated losses (5.1\%) is almost twice as much as those of other "safer" states (2.6\%-2.7\%).
} \label{tab:churnrate}
\vspace{2mm}
\begin{tabular}{|c|c|}
\hline
Last 3 Outcomes & Churn Risk                      \\ \hline
DLW $|$ LLW $|$ LDW $|$ DDD      &  2.6\% - 2.7\%        \\
... & ...  \\
WWW   &  3.7\% \\
... & ... \\
DLL $|$ LWL $|$ LDL  &  4.6\% - 4.7\%  \\
WWL & 4.9\% \\
LLL & 5.1\% \\
\hline
\end{tabular}
\end{table}

Next, we would like to study opponent recommendation to influence match outcomes of a population of pre-match players who are waiting to start matches. In PvP matches, when some players win, the other players lose. We cannot let every player win and enjoy the victory in every match. We choose the criteria of opponent recommendation as to optimize the overall engagement of the pre-match player population. A common strategy of current practical opponent recommendation systems is \textit{creating fair matches}, that is, to create matches with \textit{hard-to-predict match outcomes}. This strategy relies on the qualitative assumption that matching closely skilled players tend to create competitive environments that enhance player engagement~\cite{sweetser2005gameflow,flow1990psychology,chen2007flow,graepel2006ranking}. Such opponent recommendation systems essentially use only one dimension of personal information - player skill. However, this fundamental, yet intuitive, assumption is worthy of deep investigation. In fact, fair match outcomes may not be suitable to all types of players at all times. Consider a cautious player who cares about protecting his rank among friends, and a risk taker who enjoys difficult matches. Pairing them with the similarly skilled opponents will affect these players very differently. Even for the same player, their expectation on the coming match when they just lost three games in a row can be very different from that when they recently performed well. In Table~\ref{tab:churnrate}, we show an example that churn risks vary drastically upon players' recent match outcomes in a real-world popular PvP game. Here, churn risk, a common quantitative measurement of player engagement, is the ratio of the players who stop playing within a period time after a match. These facts lead to three key motivations: (1) it will be good if recommended opponents can complement each other's conscious or subconscious needs to engage in the game; and (2) opponent recommendation should depend on dynamic and individual \emph{player states} besides player skills. Thus, our plan is to develop an opponent recommendation framework that recommend opponents depending on player states in order to optimize the overall engagement, addressing both \textit{how} and \textit{how much} match outcomes should be affected for \textit{a population of players}.


A brief summary of this chapter is that we propose to study initial item and opponent recommendation as a way to influence PvP match outcomes. The initial item recommendation system exemplifies how match outcomes can be affected for one side of a match, while the opponent recommendation system addresses both \textit{how} and \textit{how much} match outcomes should be influenced for a population of players. The rest of the thesis will be structured as follows. Chapter~\ref{sec:relevantwork} introduces backgrounds and existing works relevant to my research. Chapter~\ref{sec:thesis} is my formal research statement and questions. Chapter~\ref{sec:plan} details my proposed algorithms to solve my research questions, including methodologies and evaluation metrics. Chapter~\ref{sec:timetable} gives a planned timeline for finishing my thesis, including what I have done and will do to complete my thesis. At last, Chapter~\ref{sec:conclusion} concludes the thesis.


