\chapter{Application} % Main chapter title

\label{chapter:application} 

This dissertation opens doors to many kinds of future applications. Match-based video games, either for serious or entertaining purposes, can benefit from using our proposed systems to boost player engagement. As a result, health games may be more effective in promoting health awareness and developing healthy habits; education games may let players absorb more knowledge; scientific problem solving games may obtain more useful solutions from players; and commercial entertaining games may engage more players and generate more profits for game companies.  

\textit{Q-DeckRec} and \textit{DraftArtist} aim to present winning-effective in-game elements to incompetent players, potentially boosting their winning probabilities and consequently improving their competence and engagement. Generally, they are best applied when a match-based game offers a space of in-game elements so large that other methods are not easily accessible or guaranteed to lead to winning-effective choices. These games stem from but are not limited to certain Collectible Card Games which require players build decks as starting items, or Multiplayer Online Battle Arena which ask players to draft character line-ups. In these two examples, the number of possibilities to build decks or the number of possible line-ups is so large that some players may feel incompetent to make winning-effective choices and approach the brink of disengagement. 

Specifically, there are several conditions for Q-DeckRec to be best applied as a winning-effective starting item recommendation system. (1) The choice space of starting items is large. (2) The winning-effectiveness of starting items can be evaluated in an acceptable time in the training phase. As in our study, we used match simulations to evaluate $f(\cdot)$, the win rate between any two decks. The time needed by $f(\cdot)$ is acceptable because our training time can be limited within 3 days, an acceptable time for offline training before the system goes online. (3) Heuristic searches do not yield good results or require domain knowledge heavy to build, and intermediate solution evaluations needed by metaheuristic searches would take up computational resources more expensive than they could afford for large-scale or real-time application (analyzed in Section~\ref{sec:qdeckrec_existmethodanaly}).

Specific conditions for DraftArtist to be best applied as a character recommendation system are as follows. (1) The choice space of characters is large. (2) The choice of characters is sequential and alternating between two teams. The sequence of character drafting is influential on the match outcome; the opponent team can take advantage of careless planning in the drafting. (3) The winning-effectiveness of a character line-up can be evaluated and can be evaluated fast. As in our study, we used a machine learning model to evaluate the win rate of a character line-up with acceptable time such that DraftArtist can output recommendation in real-time (Table~\ref{tab:mcts_time}). (4) Other methods do not yield good results. For example, association rule-based heuristics or the baseline of picking the highest win rate hero could not lead to teams as winning-effective as those formed by more sophisticated planning by DraftArtist, as shown in Table~\ref{tab:mcts} and ~\ref{tab:mcts_captain_mode}.

EOMM determines the matchmaking of opponents optimal for an engagement metric while it has  own application scopes. EOMM is not suitable for game modes meant to be played competitively and fairly, such as ranked games where players look for competing with similarly skilled players and obtaining ranks acknowledging their skill levels. In competitive game modes, skill-based matchmaking is more suitable. However, match-based games could adopt EOMM in their non-competitive modes, such as the casual mode or practice mode, to form matches that are projected to optimize an engagement metric. In our case the churn risk of all involving players (Eqn.~\ref{eqn:opt2}) is chosen as the engagement metric. As the way EOMM is designed, practitioners can plug in other engagement metrics such as play time or spending. 

% unlike authentical matchmaking, one level can be matched to multiple players
% There have also been some attempts to treat levels as players in single-player mode, in which matchmaking is applied to match players with levels as a way to adjust level difficulty for players. EOMM can also be extended in such an idea such that levels optimize


\section{Ethical Issues}\label{sec:ethical}
While our proposed systems have a broad horizon of application, we are deeply aware of many ethical issues that misuse of our systems might cause. Hence we want to raise them to all practitioners and propose several suggestions to combat them. 

% Meanwhile, we also welcome more discussion from gamers, game developers, researchers and companies, and lawmakers.

The first ethical issue is addiction. While the intent of our recommendation systems is to engage players, it remains unclear whether overuse of our systems could lead to addiction. Recent research found that high engagement seems to relate to the so-called peripheral addiction criteria (cognitive salience, tolerance, and euphoria), whereas addiction is characterized by the core addiction criteria (conflict, withdrawal symptoms, relapse and reinstatement, and behavioral salience)~\citep{charlton2007distinguishing}. Researchers have found high engagement could be a stepping stone towards addiction~\citep{charlton2007distinguishing}, while the latter is considered to be pathological, causing higher risk of psychological problems~\citep{lehenbauer2015addiction,brunborg2013gaming} and reduction in social relationships~\citep{kraut2002internet,blais2008adolescents}. To combat the potential of our systems for ``excessively'' engaging players, we suggest practitioners integrate detection mechanisms to distinguish between highly engaged and addicted players~\citep{charlton2007distinguishing,fisher1994identifying,griffiths1998dependence,brown1997theoretical,griffiths1996behavioural,seok2014distinguishing}, and adopt regulations immediately when addiction arises. Another suggestion is to regularize the optimization objective (Eqn.~\ref{eqn:opt2}) in \textit{EOMM} with some quantitative measurement of addiction level. As such, the optimization is not to simply optimize player engagement but also needs to balance with prevention of addiction.

The second issue is whether our proposed systems may become pure monetary tools of game companies for making profits, while players' game experience is sacrificed. Game companies may use winning-effective in-game recommendation described in Chapter~\ref{chapter:qdeckrec} and~\ref{chapter:draftart} with priority on profitable players - while they improve the recommendee's winning chance and potentially lead to better engagement, they may also lower the winning chance and engagement of the recommendee's opponents. Game companies may also use profit-related metrics, such as spending, as the optimization objective in the matchmaking framework described in Chapter~\ref{chapter:eomm}. To combat this issue, we would suggest more regulation on how game companies employ these systems and urge game companies reveal how their systems work in a proper manner, which leads to the next issue.

% Meanwhile, we note that no other human players' experience can be jeopardized when all players except the recommendee player are AI bots. Therefore, we suggest that practitioners should be very careful especially when integrating our systems in match-based video games involved by all human players. Models for deciding when to trigger winning-effective in-game elements or optimization objectives used in data-driven approaches should only pursue for healthy purposes of  engagement and earning not in the expense of player experience.

The third issue is how much information regarding algorithmic control introduced by our systems should be revealed to players. On one hand, game developers want to engage players in an implicit way because disclosing too much information may interrupt game experience or spoil the original intention of engaging players. On the other hand, treating our systems totally as a black box and not explaining mechanisms behind the scene may make players feel being manipulated and confused. Indeed, striking a balance on transparency is always a challenge of any Artificial Intelligence system~\citep{bostrom2014ethics,ananny2018seeing,scherer2015regulating}. So far, we think the best way for game developers and companies is to conduct honest discussion with the player community to determine the best level of transparency.

